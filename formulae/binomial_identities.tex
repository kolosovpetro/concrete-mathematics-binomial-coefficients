\documentclass[12pt,letterpaper,oneside,reqno]{amsart}
\usepackage{amsfonts}
\usepackage{amsmath}
\usepackage{amssymb}
\usepackage{amsthm}
\usepackage{float}
\usepackage{mathrsfs}
\usepackage{colonequals}
\usepackage[font=small,labelfont=bf]{caption}
\usepackage[left=1in,right=1in,bottom=1in,top=1in]{geometry}
\usepackage[pdfpagelabels,hyperindex,colorlinks=true,linkcolor=blue,urlcolor=magenta,citecolor=green]{hyperref}
\usepackage{graphicx}
\linespread{1.7}
\emergencystretch=1em
\usepackage{array}
\usepackage{etoolbox}
\apptocmd{\sloppy}{\hbadness 10000\relax}{}{}
\raggedbottom

\newtheorem{theorem}{Theorem}[section]
\newtheorem{corollary}[theorem]{Corollary}
\newtheorem{lemma}[theorem]{Lemma}
\newtheorem{example}[theorem]{Example}
\newtheorem{conjecture}[theorem]{Conjecture}
\newtheorem{definition}[theorem]{Definition}

\numberwithin{equation}{section}

\title[Binomial identities]
{Binomial identities}
\author[Petro Kolosov]{Petro Kolosov}

\begin{document}

    \maketitle

    \section{Binomial identities}\label{sec:binomial-identities}
    Identity for negative $r$ in binomial coefficients
    \begin{align*}
        \binom{r}{k}  &= (-1)^k \binom{k-r-1}{k} \\
        \binom{-r}{k} &= (-1)^k \binom{k+r-1}{k}
    \end{align*}
    Thus, the generating function follows
    \begin{align*}
        \frac{1}{(1+z)^r} &= \sum_{k} (-1)^k \binom{k+r-1}{k} z^k = 1 - \binom{r}{1}x + \binom{r+1}{2} x^2 - \binom{r+2}{3} x^3 \cdots \\
        \frac{1}{(1-z)^r} &= \sum_{k} \binom{k+r-1}{k} z^k = 1 + \binom{r}{1}x + \binom{r+1}{2} x^2 + \binom{r+2}{3} x^3 \cdots
    \end{align*}
    Thus
    \begin{align*}
        [z^n] \frac{1}{(1-z)^r} &= \binom{n+r-1}{n}
    \end{align*}
    Cauchy product of two generating functions
    \begin{align*}
        A(z) \cdot B(z) = \sum_{n=0}^{\infty} \left( \sum_{k=0}^{n} a_k b_{n-k} \right) z^n
    \end{align*}
\end{document}
