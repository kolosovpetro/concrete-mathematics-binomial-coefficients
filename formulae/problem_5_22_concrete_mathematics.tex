\documentclass{article}
\usepackage{amsmath}
\usepackage{amssymb}
\usepackage{hyperref}

\begin{document}
    \section{The problem}\label{sec:the-problem}
    Use generating functions to prove that
    \begin{equation}
        \sum_{k} \binom{r}{m+k} \binom{s}{n-k} = \binom{r+s}{m+n}
        \label{eq:identity-to-prove}
    \end{equation}
    Okay, first let's review the summation boundary such that terms are non-zero.
    Summation is done over $k$ so that binomial coefficient $\binom{s}{n-k}$ fixes $k$ to be less or equal to $n$.
    Rewrite the statement of the problem
    \begin{equation}
        \sum_{k=0}^{n} \binom{r}{m+k} \binom{s}{n-k} = \binom{r+s}{m+n}
        \label{eq:identity-to-prove-2}
    \end{equation}
    Left-hand side of it reminds me sequence convolution of two generating functions.
    Let be two generating functions for such left-hand side summation:
    \begin{equation*}
        A_r(z); \quad B_s(z)
    \end{equation*}
    Multiplying those generating functions yields
    \begin{align*}
        C(x) = \left(\sum_{m=0}^{\infty} a_m x^m\right) \left(\sum_{n=0}^{\infty} b_n x^n\right) = \sum_{k=0}^{\infty} \left( \sum_{m=0}^{k} a_m b_{k-m} \right) x^k
    \end{align*}
    Then
    \begin{align*}
        A_r(z) \cdot B_s(z) = \sum_{n=0}^{\infty} \left( \sum_{k=0}^{n} a_k b_{n-k} \right) x^n
    \end{align*}
    We can notice the similar structure as we have in our problem~\eqref{eq:identity-to-prove-2}.
    So let's find the generating function for the binomial coefficient $\binom{r+s}{m+n}$.
    We know that generating function for the binomial coefficient $\binom{n}{k}$ is
    \begin{align*}
    (1+z)
        ^{n} = \sum_{k=0}^{\infty} \binom{n}{k} z^k
    \end{align*}
    If we want to have $m+k$ as lower index, then
    \begin{align*}
    (1+z)
        ^{r} = \sum_{k=0}^{\infty} \binom{r}{m+k} z^{m+k} \\
        (1+z)^{r} = z^m \sum_{k=0}^{\infty} \binom{r}{m+k} z^{k} \\
        \frac{(1+z)^{r}}{z^m} = \sum_{k=0}^{\infty} \binom{r}{m+k} z^{k}
    \end{align*}
    Thus, the coefficient of $z^n$ in $\frac{(1+z)^{r}}{z^m}$ is
    \begin{align*}
    [z^n]
        \frac{(1+z)^{r}}{z^m} = \binom{r}{m+n}
    \end{align*}
    So that our first generating function is
    \begin{align*}
        A_r(z) = \frac{(1+z)^{r}}{z^m}
    \end{align*}
    The second generating function is
    \begin{align*}
        B_s(z) = (1+z)^{s}
    \end{align*}
    Multiplying them
    \begin{align*}
        A_r(z) \cdot B_s(z) = \frac{(1+z)^{r}}{z^m} \cdot (1+z)^{s} = \frac{(1+z)^{r+s}}{z^m}
    \end{align*}
    Convolution form is
    \begin{align*}
        A_r(z) \cdot B_s(z) = \sum_{n=0}^{\infty} \left( \sum_{k=0}^{n} a_k b_{n-k} \right) x^n \\
        = \sum_{n=0}^{\infty} \left( \sum_{k=0}^{n} \binom{r}{m+k} \binom{s}{n-k} \right) x^n
    \end{align*}
    Coefficient of $z^n$ in $\frac{(1+z)^{r+s}}{z^m}$ is
    \begin{align*}
    [z^n]
        \frac{(1+z)^{r+s}}{z^m} = \binom{r+s}{m+n}
    \end{align*}
    Coefficient of $z^n$ in $\sum_{k=0}^{\infty} \left( \sum_{k=0}^{n} \binom{r}{m+k} \binom{s}{n-k} \right) x^n$ is
    \begin{align*}
    [z^n]
        A_r(z) \cdot B_s(z) = \sum_{k=0}^{n} \binom{r}{m+k} \binom{s}{n-k}
    \end{align*}
\end{document}
