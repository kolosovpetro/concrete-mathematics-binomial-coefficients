\documentclass[12pt,letterpaper,oneside,reqno]{amsart}
\usepackage{amsfonts}
\usepackage{amsmath}
\usepackage{amssymb}
\usepackage{amsthm}
\usepackage{float}
\usepackage{mathrsfs}
\usepackage{colonequals}
\usepackage[font=small,labelfont=bf]{caption}
\usepackage[left=1in,right=1in,bottom=1in,top=1in]{geometry}
\usepackage[pdfpagelabels,hyperindex,colorlinks=true,linkcolor=blue,urlcolor=magenta,citecolor=green]{hyperref}
\usepackage{graphicx}
\linespread{1.7}
\emergencystretch=1em
\usepackage{array}

\newtheorem{theorem}{Theorem}[section]
\newtheorem{corollary}[theorem]{Corollary}
\newtheorem{lemma}[theorem]{Lemma}
\newtheorem{example}[theorem]{Example}
\newtheorem{conjecture}[theorem]{Conjecture}
\newtheorem{definition}[theorem]{Definition}
\newtheorem{identity}[theorem]{Identity}

\newcommand \coefficientOf [2] {[#1]^{#2}}

\numberwithin{equation}{section}

\title[Comments on Concrete Mathematics (2e) Binomial coefficients]
{Comments on Concrete Mathematics (2e) Binomial coefficients}
\author[Petro Kolosov]{Petro Kolosov}
\begin{document}

    \maketitle

    \tableofcontents


    \section{Conventions}\label{sec:conventions}
    \begin{itemize}
    \item Use variable $z$ that indicates complex value in generating functions.
    \item Give particular names to binomial identities, for example \textit{absorption identity}
    \item Give particular names to generating functions to remember them easily
\end{itemize}



    \section{Important binomial identities}\label{sec:important-binomial-identities}
    Identities from Concrete Mathematics~\cite[p. 174]{graham1994concrete}
\begin{identity}
    Factorial expansion:
    \begin{align*}
        \binom{n}{k} &= \frac{n!}{k!(n-k)!}, \quad \text{integers } n \geq k \geq 0.
    \end{align*}
\end{identity}

\begin{identity}
    Symmetry:
    \begin{align*}
        \binom{n}{k} &= \binom{n}{n-k}, \quad \text{integer } n \geq 0, \text{ integer } k.
    \end{align*}
\end{identity}

\begin{identity}
    Absorption/extraction:
    \begin{align*}
        \binom{r}{k} &= \frac{r}{k} \binom{r-1}{k-1}, \quad \text{integer } k \neq 0. \\
        k \binom{r}{k} &= r \binom{r-1}{k-1} \\
        (r-k) \binom{r}{k} &= (r-k) \binom{r}{r-k} = r \binom{r-1}{r-k-1} = r \binom{r-1}{k}.
    \end{align*}
\end{identity}

\begin{identity}
    Addition/induction:
    \begin{align*}
        \binom{r}{k} &= \binom{r-1}{k} + \binom{r-1}{k-1}, \quad \text{integer } k.
    \end{align*}
\end{identity}

\begin{identity}
    Upper negation:
    \begin{align*}
        \binom{r}{k} &= (-1)^k \binom{k - r - 1}{k}, \quad \text{integer } k.
    \end{align*}
    Let $r=\ell-1-t$ and $k=\ell-t-m+s$, then
    \begin{align*}
        \binom{\ell-1-t}{\ell-t-m+s} &= (-1)^{\ell-t-m+s} \binom{m+s}{\ell-t-m+s}.
    \end{align*}
    Also
    \begin{align*}
    (-1)
        ^{t+s} \binom{\ell-1-t}{\ell-t-m+s}
        &= (-1)^{\ell-m+2s} \binom{m+s}{\ell-t-m+s} \\
        &= (-1)^{\ell-m} \binom{m+s}{\ell-t-m+s} \\
        &= (-1)^{\ell+m} \binom{m+s}{\ell-t-m+s}
    \end{align*}
    Because
    \begin{align*}
    (-1)
        ^{k+2s} \binom{n}{t} &= (-1)^k \binom{n}{t} \\
        (-1)^{k-s} \binom{n}{t}  &= (-1)^{k+s} \binom{n}{t}
    \end{align*}
\end{identity}

\begin{identity}
    Trinomial revision:
    \begin{align*}
        \binom{r}{m} \binom{m}{k} &= \binom{r}{k} \binom{r - k}{m - k}, \quad \text{integers } m, k.
    \end{align*}
\end{identity}

\begin{identity}
    Binomial theorem:
    \begin{align*}
        \sum_{k} \binom{r}{k} x^k y^{r-k} &= (x + y)^r, \quad \text{integer } r \geq 0, \text{ or } |x/y| < 1.
    \end{align*}
\end{identity}

\begin{identity}
    Parallel summation:
    \begin{align*}
        \sum_{k \leq n} \binom{r + k}{k} &= \binom{r + n + 1}{n}, \quad \text{integer } n.
    \end{align*}
\end{identity}

\begin{identity}
    Upper summation:
    \begin{align*}
        \sum_{0 \leq k \leq n} \binom{k}{m} &= \binom{n + 1}{m + 1}, \quad \text{integers } m, n \geq 0.
    \end{align*}
\end{identity}

\begin{identity}
    Vandermonde convolution:
    \begin{align*}
        \sum_{k} \binom{r}{k} \binom{s}{n-k} &= \binom{r + s}{n}, \quad \text{integer } n.
    \end{align*}
\end{identity}




\clearpage



    \section{Important generating functions}\label{sec:important-generating-functions}
    \begin{identity}
    Cauchy product rule of two generating functions $A(z), \; B(z)$
    \begin{align*}
        A(z) \cdot B(z) = \left( \sum_{n=0}^{\infty} a_n z^n \right) \left( \sum_{n=0}^{\infty} b_n z^n \right)
        = \sum_{n=0}^{\infty} \left( \sum_{k=0}^{n} a_k b_{n-k} \right) z^n
    \end{align*}
\end{identity}
\begin{identity}
    Cauchy product rule for $(1+z)^{r+s}$
    \begin{align*}
    (1 + z)
        ^{r+s} = \sum_{n=0}^{\infty} \left( \sum_{k=0}^{n} \binom{r}{k} \binom{s}{n-k} \right) z^n
    \end{align*}
\end{identity}
\begin{identity}
    Shift selected coefficient of generating function
    \begin{align*}
    [z^{p-q}]
        A(z)=[z^p] z^{q} A(z) \\
        [z^{p+q}] A(z)=[z^p] \frac{1}{z^{q}} A(z)
    \end{align*}
\end{identity}
\begin{identity}
    Binomial coefficient, fixed $r$
    \begin{align*}
        \binom{r}{n} = \coefficientOf{z}{n} (1+z)^r
    \end{align*}
\end{identity}
\begin{identity}
    Shifted binomial coefficient, fixed $m, r$
    \begin{align*}
        \binom{r}{m+n} = \coefficientOf{z}{n} \frac{(1+z)^r}{z^m}
    \end{align*}
\end{identity}
\begin{identity}
    Binomial coefficient of multiset~\cite[eq. 8]{faris2011generating}, fixed k
    \begin{align*}
        A_k(z) = \sum_{n=0}^{\infty} \binom{n}{k} z^n = \frac{z^k}{(1-z)^{k+1}}
    \end{align*}
    Then
    \begin{align*}
        \binom{t}{k} = \coefficientOf{z}{t} \frac{z^k}{(1-z)^{k+1}}
    \end{align*}
    So that iteration goes over upper index of binomial coefficient.
\end{identity}
\begin{identity}
    Shifted Binomial coefficient of multiset, fixed k
    \begin{align*}
        \binom{t}{k+r} = \coefficientOf{z}{t} \frac{z^{k+r}}{(1-z)^{k+r+1}}
    \end{align*}
\end{identity}
\begin{identity}
    Shifted Binomial coefficient of multiset in two variables~\cite[eq. 15]{faris2011generating}
    \begin{align*}
        B(x, y) = \sum_{n=0}^{\infty} \sum_{k=0}^{\infty} \binom{n}{k} x^k y^n
        = \sum_{n=0}^{\infty} (1 + x)^n y^n = \frac{1}{1 - (1 + x)y}
    \end{align*}
\end{identity}
\begin{identity}
    Shifted Binomial coefficient of multiset in two variables (negated)
    \begin{align*}
        \sum_{n=0}^{\infty} (1 + x)^n y^n (-1)^n = \frac{1}{1 + (1 + x)y}
    \end{align*}
\end{identity}
\begin{identity}
    Binomial coefficients row summation East-West
    \begin{align*}
        \sum_{k} \binom{r}{j-k}
        &= \sum_{k} \coefficientOf{z}{j-k}(1+z)^r
        = \coefficientOf{z}{j} \sum_{k} z^k (1+z)^r \\
        &= (1+z)^r \coefficientOf{z}{j} \sum_{k} z^k
    \end{align*}
    Because $[z^{p-q}] A(z)=[z^p] z^{q} A(z)$
\end{identity}




\clearpage



    \section{Important binomial sums}\label{sec:important-binomial-sums}
    Identities from Concrete Mathematics~\cite[p. 169]{graham1994concrete}
\begin{identity}
    \begin{align*}
        \sum_{k} \binom{r}{m+k} \binom{s}{n-k} &= \binom{r+s}{m+n}, \quad \text{integers } m, n.
    \end{align*}
\end{identity}

\begin{identity}
    \begin{align*}
        \sum_{k} \binom{l}{m+k} \binom{s}{n+k} &= \binom{l+s}{l-m+n}, \quad \text{integer } l \geq 0, \text{integers } m, n.
    \end{align*}
\end{identity}

\begin{identity}
    \begin{align*}
        \sum_{k} \binom{l}{m+k} \binom{s+k}{n} (-1)^k &= (-1)^{l+m} \binom{s-m}{n-l}, \quad \text{integer } l \geq 0, \text{integers } m, n.
    \end{align*}
\end{identity}

\begin{identity}
    \begin{align*}
        \sum_{k \leq l} \binom{l-k}{m} \binom{s}{k-n} (-1)^k &= (-1)^{l+m} \binom{s-m-1}{l-m-n}, \quad \text{integers } l, m, n \geq 0.
    \end{align*}
\end{identity}

\begin{identity}
    \begin{align*}
        \sum_{0 \leq k \leq l} \binom{l-k}{m} \binom{q+k}{n} &= \binom{l+q+1}{m+n+1}, \quad \text{integers } l, m \geq 0, \text{integers } n \geq q \geq 0.
    \end{align*}
\end{identity}


    \section{Problem 1: Prove that $\sum_{k=0}^{t} \binom{t-k}{r} \binom{k}{s} = \binom{t+1}{r+s+1}$}
\label{sec:problem-1-column-summation-of-two-binomial-coefficients-mse}
Prove that
\begin{equation}
    \sum_{k=0}^{t} \binom{t-k}{r} \binom{k}{s} = \binom{t+1}{r+s+1}
    \label{eq:identity-to-prove-problem-1}
\end{equation}

We can see that iteration in the left-hand side of the equation~\eqref{eq:identity-to-prove-problem-1} is running
over the upper index of binomial coefficients, so let's figure out proper generating function for it.
We know that
\begin{equation}
    \sum_{n=0}^{\infty} \binom{n}{k} y^n = \frac{y^k}{(1-y)^{k+1}}\label{eq:generating-function-base}
\end{equation}
Now we keep attention to the lower index of binomial coefficient in right-hand side
of the equation~\eqref{eq:identity-to-prove-problem-1} which is $r+s+1$.
We have to match some generating function to reach the $\sum_{n=0}^{\infty} \binom{n}{r+s+1} x^n$
Let be generating functions
\begin{align*}
    A_r(x) &= \sum_{l=0}^{\infty} \binom{l}{r} x^l\\
    B_s(x) &= \sum_{k=0}^{\infty} \binom{k}{s} x^k
\end{align*}
Now let's match our generating function
\begin{align*}
    x A_r(x) B_s(x)
    &= x \cdot \left( \sum_{l=0}^{\infty} \binom{l}{r} x^l \right) \cdot \left( \sum_{k=0}^{\infty} \binom{k}{s} x^k \right) \\
    &= x \cdot \frac{x^r}{(1-x)^{r+1}} \cdot \frac{x^s}{(1-x)^{s+1}} = \frac{x^{r+s+1}}{(1-x)^{r+s+2}} \\
    &= \sum_{n=0}^{\infty} \binom{n}{r+s+1} x^n
\end{align*}
Because of the equation~\eqref{eq:generating-function-base}.
Actually, we could simply substitute $k=r+s+1$ to the equation~\eqref{eq:generating-function-base} to reach desired
generating function, anyway.

The coefficient of $x^{t+1}$ in the $x \cdot \sum_{l=0}^{\infty} \binom{l}{r} x^l \cdot \sum_{k=0}^{\infty} \binom{k}{s} x^k$ is
\begin{align*}
[x^{t+1}]
    x A_r(x) B_s(x) = \sum_{k=0}^{t} a_k b_{t-k} = \sum_{k=0}^{t} \binom{t-k}{r} \binom{k}{s}
\end{align*}
Note that upper summation bound is $t$ while coefficient is $[z^{t+1}]$ it is because of the $x$ factor in the generating function.
General rule for that is
\begin{align*}
[z^{p-q}]
    A(z)=[z^p]z^qA(z)
\end{align*}
While
\begin{align*}
[x^{t+1}]
    \sum_{n=0}^{\infty} \binom{n}{r+s+1} x^n = \binom{t+1}{r+s+1}
\end{align*}
Thus
\begin{align*}
    \binom{t+1}{r+s+1} = \sum_{k=0}^{t} \binom{t-k}{r} \binom{k}{s}
\end{align*}

This approach is based on generating functions
by \url{http://math.arizona.edu/~faris/combinatoricsweb/generate.pdf}

\subsection{Flow of the solution}
\label{subsec:flow-of-the-solution}
\begin{itemize}
    \item First, keep your attention to the left part of the problem~\eqref{eq:identity-to-prove-problem-1}
    to see over what index iteration is running, there can be three cases: lower, upper, or both.
    If both find identity to simplify it to run over either lower or upper index.
    If is upper index use generating function~\eqref{eq:generating-function-base}.
    If lower index use binomial theorem as generating function.
    \item Then keep your attention to the right part of the problem~\eqref{eq:identity-to-prove-problem-1},
    precisely the lower index of binomial coefficient, as it defines the generating function
    we're looking for.
    \item Then seeing the sum over binomial coefficients multiplication,
    it means that it should be expressed in terms of convolution of two generating functions,
    for our case is $[x^{t+1}] x A_r(x) B_s(x)$.
    \item Match the coefficient given by the convolution of two generating functions and the generating function.
\end{itemize}


\clearpage


    \section{Problem 2: Concrete mathematics equation (5.22)}
\label{sec:problem-2:-concrete-mathematics-equation-(5.22)}
Use generating functions to prove that
\begin{equation}
    \sum_{k} \binom{r}{m+k} \binom{s}{n-k} = \binom{r+s}{m+n}
    \label{eq:identity-to-prove}
\end{equation}
Okay, first let's review the summation boundary such that terms are non-zero.
Summation is done over $k$ so that binomial coefficient $\binom{s}{n-k}$ fixes $k$ to be less or equal to $n$.
Rewrite the statement of the problem
\begin{equation}
    \sum_{k=0}^{n} \binom{r}{m+k} \binom{s}{n-k} = \binom{r+s}{m+n}
    \label{eq:identity-to-prove-2}
\end{equation}
Left-hand side of it reminds me sequence convolution of two generating functions.
Let be two generating functions for such left-hand side summation:
\begin{equation*}
    A_r(z); \quad B_s(z)
\end{equation*}
Multiplying those generating functions yields
\begin{align*}
    C(x) = \left(\sum_{m=0}^{\infty} a_m x^m\right) \left(\sum_{n=0}^{\infty} b_n x^n\right) = \sum_{k=0}^{\infty} \left( \sum_{m=0}^{k} a_m b_{k-m} \right) x^k
\end{align*}
Then
\begin{align*}
    A_r(z) \cdot B_s(z) = \sum_{n=0}^{\infty} \left( \sum_{k=0}^{n} a_k b_{n-k} \right) x^n
\end{align*}
We can notice the similar structure as we have in our problem~\eqref{eq:identity-to-prove-2}.
So let's find the generating function for the binomial coefficient $\binom{r+s}{m+n}$.
We know that generating function for the binomial coefficient $\binom{n}{k}$ is
\begin{align*}
(1+z)
    ^{n} = \sum_{k=0}^{\infty} \binom{n}{k} z^k
\end{align*}
If we want to have $m+k$ as lower index, then
\begin{align*}
(1+z)
    ^{r} = \sum_{k=0}^{\infty} \binom{r}{m+k} z^{m+k} \\
    (1+z)^{r} = z^m \sum_{k=0}^{\infty} \binom{r}{m+k} z^{k} \\
    \frac{(1+z)^{r}}{z^m} = \sum_{k=0}^{\infty} \binom{r}{m+k} z^{k}
\end{align*}
Thus, the coefficient of $z^n$ in $\frac{(1+z)^{r}}{z^m}$ is
\begin{align*}
[z^n]
    \frac{(1+z)^{r}}{z^m} = \binom{r}{m+n}
\end{align*}
So that our first generating function is
\begin{align*}
    A_r(z) = \frac{(1+z)^{r}}{z^m}
\end{align*}
The second generating function is
\begin{align*}
    B_s(z) = (1+z)^{s}
\end{align*}
Multiplying them
\begin{align*}
    A_r(z) \cdot B_s(z) = \frac{(1+z)^{r}}{z^m} \cdot (1+z)^{s} = \frac{(1+z)^{r+s}}{z^m}
\end{align*}
Convolution form is
\begin{align*}
    A_r(z) \cdot B_s(z) = \sum_{n=0}^{\infty} \left( \sum_{k=0}^{n} a_k b_{n-k} \right) x^n \\
    = \sum_{n=0}^{\infty} \left( \sum_{k=0}^{n} \binom{r}{m+k} \binom{s}{n-k} \right) x^n
\end{align*}
Coefficient of $z^n$ in $\frac{(1+z)^{r+s}}{z^m}$ is
\begin{align*}
[z^n]
    \frac{(1+z)^{r+s}}{z^m} = \binom{r+s}{m+n}
\end{align*}
Coefficient of $z^n$ in $\sum_{k=0}^{\infty} \left( \sum_{k=0}^{n} \binom{r}{m+k} \binom{s}{n-k} \right) x^n$ is
\begin{align*}
[z^n]
    A_r(z) \cdot B_s(z) = \sum_{k=0}^{n} \binom{r}{m+k} \binom{s}{n-k}
\end{align*}

\clearpage


    \section{Problem 3: Prove that $\sum_{k} \binom{l}{m+k} \binom{s}{n+k} = \binom{l+s}{l-m+n}$}
\label{sec:problem-3:-concrete-mathematics-equation-(5.23)}
This is from Concrete mathematics equation (5.23).
Use generating functions to prove that
\begin{equation}
    \sum_{k} \binom{l}{m+k} \binom{s}{n+k} = \binom{l+s}{l-m+n}
    \label{eq:identity-to-prove-problem-3}
\end{equation}
The first problem is that it is not clear what summation bounds should be to keep non-zero terms.
Let's reverse the coefficient $\binom{l}{m+k}$ to see exact summation bounds
\begin{align*}
    \sum_{k=0}^{l-m} \binom{l}{l-m-k} \binom{s}{n+k} = \binom{l+s}{l-m+n}
\end{align*}
So now it is clear that we are hunting for the coefficient of $z^{l-m}$ in the generating function.
Let's rearrange the sum above
\begin{align*}
    \sum_{k=0}^{(l-m)} \binom{s}{n+k} \binom{l}{(l-m)-k} = \binom{l+s}{n+(l-m)}
\end{align*}
So we can see that left-hand side of the equation above matches the convolution of two generating functions.
\begin{align*}
    A(z) \cdot B(z) = \sum_{n=0}^{\infty} \left( \sum_{k=0}^{n} a_k b_{n-k} \right) z^n
\end{align*}
having $n=l-m$ so that basically our identity is taking the coefficient of $z^{l-m}$
in the convolution of two generating functions.
Let be $n=t$ for the sake of simplicity
\begin{equation}
    \sum_{k=0}^{(l-m)} \binom{s}{t+k} \binom{l}{(l-m)-k} = \binom{l+s}{t+(l-m)}
    \label{eq:identity-to-prove-problem-3-2}
\end{equation}
So we have to match two generating functions $A_s(z), \; B_l(z)$ for
the binomial coefficients: $\binom{s}{t+k}$ and $\binom{l}{(l-m)-k}$.
So that
\begin{align*}
    A_s(z) = \frac{(1+z)^s}{z^t} \\
    B_l(z) = (1+z)^l \\
\end{align*}
So that product of them
\begin{align*}
    A_s(z) \cdot B_l(z) = \sum_{n=0}^{\infty} \left( \sum_{k=0}^{n} \binom{s}{t+k} \binom{l}{n-k} \right) z^n
    = \frac{(1+z)^s}{z^t} \cdot (1+z)^l = \frac{(1+z)^{l+s}}{z^t}
\end{align*}
Right side of the~\eqref{eq:identity-to-prove-problem-3-2} is the coefficient of
$z^{l-m}$ in $\frac{(1+z)^{l+s}}{z^t}$
\begin{align*}
[z^{l-m}]
    \frac{(1+z)^{l+s}}{z^t} = \binom{l+s}{l-m+t}
\end{align*}
Left side of the~\eqref{eq:identity-to-prove-problem-3-2} is the coefficient of $z^{l-m}$
in the convolution of two generating functions
\begin{align*}
[z^{l-m}]
    A_s(z) \cdot B_l(z) = \sum_{k=0}^{(l-m)} \binom{s}{t+k} \binom{l}{(l-m)-k}
\end{align*}
Thus,
\begin{align*}
    \sum_{k=0}^{(l-m)} \binom{s}{t+k} \binom{l}{(l-m)-k} = \binom{l+s}{l-m+t}
\end{align*}

\clearpage


    \section{Problem 4: Prove that $\sum_{j} \binom{n}{j}^2 = \binom{2n}{n}$}
\label{sec:problem-4}
Using generating function prove that
\begin{align*}
    \sum_{j} \binom{n}{j}^2 = \binom{2n}{n}
\end{align*}

\clearpage


    \section{Problem 5: Prove that $\sum_{k \leq n} \binom{r+k}{k} = \binom{r+n+1}{n}$}
\label{sec:problem-5:-concrete-mathematics-5.9}


    \section{Problem 6: Prove that $\sum_{k \leq n} \binom{k}{m} = \binom{n+1}{m+1}$}
\label{sec:problem-6}
This is from Concrete Mathematics equation (5.10).
Using generating function prove that
\begin{align}
    \sum_{k=0}^{n} \binom{k}{m} = \binom{n+1}{m+1}
    \label{eq:identity-to-prove-problem-6}
\end{align}
Now it is clear that we are looking for the coefficient of $z^n$ in the convolution of two generating functions.
Rewrite problem to match the Cauchy product rule
\begin{align}
    \sum_{k=0}^{n} \binom{k}{m} \binom{n-k}{j} = \binom{n+1}{m+1}
    \label{eq:problem-6-general-form}
\end{align}
Generating function for the binomial coefficient $\binom{k}{m}$ is given by
\begin{align*}
    A_m(z) = \sum_{k=0}^{\infty} \binom{k}{m} z^k = \frac{z^m}{(1-z)^{m+1}}
\end{align*}
Generating function for the binomial coefficient $\binom{n-k}{j}$ is given by
\begin{align*}
    B_j(z) = \sum_{k=0}^{\infty} \binom{k}{j} z^k = \frac{z^j}{(1-z)^{j+1}}
\end{align*}
Multiplication of them yields
\begin{align*}
    A_m(z) \cdot B_j(z)
    &= \sum_{n=0}^{\infty} \left( \sum_{k=0}^{n} \binom{k}{m} \binom{n-k}{j} \right) z^n
    = \frac{z^m}{(1-z)^{m+1}} \cdot \frac{z^j}{(1-z)^{j+1}} \\
    &= \frac{z^{m+j}}{(1-z)^{m+j+2}}
\end{align*}
Let $j=0$ then
\begin{align*}
    A_m(z) \cdot B_0(z) = \frac{z^m}{(1-z)^{m+1}} \cdot \frac{1}{(1-z)} = \frac{z^m}{(1-z)^{m+2}}
\end{align*}
Note that generating function above does not match the generating function of binomial coefficient $\binom{k}{n}$,
fixed $n$.
Multiplying by $z$ gives
\begin{align*}
    z A_m(z) \cdot B_0(z)
    &= \sum_{n=0}^{\infty} \left( \sum_{k=0}^{n} \binom{k}{m} \binom{n-k}{0} \right) z^{n+1}
    = \frac{z^m}{(1-z)^{m+1}} \cdot \frac{1}{(1-z)} \\
    = \frac{z^{m+1}}{(1-z)^{m+2}}
\end{align*}
Now it matches.
Taking the coefficient of $z^{n+1}$ yields
\begin{align*}
    [z^{n+1}] z A_m(z) \cdot B_0(z) = \binom{n+1}{m+1}
\end{align*}
Therefore, it is indeed true that
\begin{align*}
    \sum_{k=0}^{n} \binom{k}{m} = \binom{n+1}{m+1}
\end{align*}
\clearpage


    \bibliographystyle{unsrt}
    \bibliography{ConcreteMathematicsComments}
    \noindent \textbf{Version:} \texttt{Local-0.1.0}

\end{document}
