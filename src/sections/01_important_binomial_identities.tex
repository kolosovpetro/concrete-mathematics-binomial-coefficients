\subsection{Generating functions}\label{subsec:generating-functions}
Generating function is a power series that generates an infinite sequence of numbers
$\{a_0, a_1, a_2, a_3, \dots \}$
\begin{align*}
    A(z) = a_0 + a_1 z + a_2 z^2 + a_3 z^3 + \dots = \sum_{k=0}^{\infty} a_k z^k
\end{align*}
Coefficient of $z^n$ in $A(z)$ denoted as
\begin{align*}
[z^n]
    A(z) = a_n
\end{align*}

is the $n$-th term of the sequence.
For example, generating function for the sequence of binomial coefficients is
\begin{align*}
    (1+z)^r = \sum_{k=0}^{\infty} \binom{r}{k} z^k
\end{align*}
Let be a product of two generating functions $A(z)$ and $B(z)$, then $c_n$ in such sequence is a sum
\begin{align*}
    c_n = \sum_{k=0}^{n} a_k b_{n-k}
\end{align*}
Above sum is called the convolution of two sequences $\{a_0, a_1, a_2, a_3, \dots \}$ and $\{b_0, b_1, b_2, b_3, \dots \}$.
So that
\begin{align*}
[z^n]
    A(z) B(z) = c_n
\end{align*}
Example for Vandermonde convolution, let be $A(z)=(1+z)^r$ and $B(z)=(1+z)^s$, then multiplying them
\begin{align*}
    A(z) B(z) = (1+z)^{r} (1+z)^{s} = (1+z)^{r+s}
\end{align*}
Then the coefficient of $z^n$ in $(1+z)^{r+s}$ is
\begin{align*}
[z^n]
    A(z) B(z) = [z^n] (1+z)^{r+s} = \sum_{k=0}^{n} a_k b_{n-k}= \sum_{k=0}^{n} \binom{r}{k} \binom{s}{n-k} = \binom{r+s}{n}
\end{align*}
