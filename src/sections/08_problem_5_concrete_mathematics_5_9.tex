\section{Problem 5: Prove that $\sum_{k \leq n} \binom{r+k}{k} = \binom{r+n+1}{n}$}
\label{sec:problem-5:-concrete-mathematics-5.9}
This is from Concrete Mathematics equation (5.9).
Using generating function prove that
\begin{align}
    \sum_{k=0}^{n} \binom{r+k}{k} = \binom{r+n+1}{n} = \binom{r+n+1}{r+1}
    \label{eq:identity-to-prove-problem-5}
\end{align}
So we are looking for the coefficient of $z^n$ in multiplication
of two yet-unspecified generating functions $A, B$ because of upper index of summation $n$.

\subsection{Left-hand side of the identity}\label{subsec:left-hand-side-of-the-identity}
Rewrite our identity to get rid of $k$ as lower index of binomial coefficient,
that makes easier to guess required generating function.
By symmetry of binomial coefficients we have
\begin{align*}
    \sum_{k=0}^{n} \binom{r+k}{r} = \binom{r+n+1}{n}
\end{align*}
Multiply the sum by $\binom{n-k}{j}$ to match the Cauchy product rule
\begin{align*}
    \sum_{k=0}^{n} \binom{r+k}{r} \binom{n-k}{j}
\end{align*}
As iteration goes over upper index of binomial coefficients, let be $r=l$ in
lower index of $\binom{r+k}{r}$.
So for fixed $r, l, j$ we have general form of the problem
\begin{equation}
    \sum_{k=0}^{n} \binom{r+k}{l} \binom{n-k}{j}
    \label{eq:problem-5-general-form}
\end{equation}
Looking to the upper index of summation $n$ we can conclude that we are looking for
the coefficient of $z^n$ in the convolution of two generating functions in the
Cauchy product rule of two generating functions $A(z), \; B(z)$
\begin{align*}
    A(z) \cdot B(z) = \left( \sum_{n=0}^{\infty} a_n z^n \right) \left( \sum_{n=0}^{\infty} b_n z^n \right)
    = \sum_{n=0}^{\infty} \left( \sum_{k=0}^{n} a_k b_{n-k} \right) z^n
\end{align*}
As summation runs over upper index of binomial coefficients, our basic generating function is
\begin{align*}
    \binom{t}{k} = \coefficientOf{z}{t} \frac{z^k}{(1-z)^{k+1}}
\end{align*}
So that
\begin{align*}
    \coefficientOf{z}{t} \sum_{m=0}^{\infty} \binom{m}{k} z^m = \coefficientOf{z}{t} \frac{z^k}{(1-z)^{k+1}}
\end{align*}
Consider the binomial coefficient $\binom{r+n}{l}$ and its generating function.
Note that $r$ is fixed value there and iteration goes over $n$.
Therefore,
\begin{align*}
    \coefficientOf{z}{t} \sum_{m=0}^{\infty} \binom{r+m}{k} z^{r+m} &= \coefficientOf{z}{t} \frac{z^k}{(1-z)^{k+1}} \\
    \coefficientOf{z}{t} \sum_{m=0}^{\infty} \binom{r+m}{k} z^{m}   &= \coefficientOf{z}{t} \frac{z^k}{(1-z)^{k+1} \cdot z^r}
\end{align*}
Denote above generating function as $A_{r,k}(z)$.
\begin{align*}
    A_{r,k}(z) = \sum_{m=0}^{\infty} \binom{r+m}{k} z^{m} = \frac{z^k}{(1-z)^{k+1} \cdot z^r}
\end{align*}
Consider the second binomial coefficient $\binom{n-k}{j}$ where $j$ is fixed value.
Its generating function $B_j(z)$ is
\begin{align*}
    B_j(z) = \frac{z^j}{(1-z)^{j+1}}
\end{align*}
So considering base form of the problem~\eqref{eq:problem-5-general-form} we have the following generating functions.
For $\binom{r+k}{l}$ having fixed $r, l$
\begin{align*}
    A_{r,l}(z) = \sum_{m=0}^{\infty} \binom{r+m}{l} z^{m} = \frac{z^l}{(1-z)^{l+1} \cdot z^r}
\end{align*}
for $\binom{n-k}{j}$ having fixed $j$
\begin{align*}
    B_j(z) = \frac{z^j}{(1-z)^{j+1}}
\end{align*}
Multiplication yields
\begin{align*}
    A_{r,l}(z) \cdot B_j(z) &= \frac{z^l}{(1-z)^{l+1} \cdot z^r} \cdot \frac{z^j}{(1-z)^{j+1}}
    = \frac{z^{l+j}}{(1-z)^{l+j+2} \cdot z^r} \\
    &= \frac{z^{l+j-r}}{(1-z)^{l+j+2}}
\end{align*}
General Cauchy form yields
\begin{align*}
    A_{r,l}(z) \cdot B_j(z) &= \sum_{n=0}^{\infty} \left( \sum_{k=0}^{n} a_k b_{n-k} \right) z^n \\
    &= \sum_{n=0}^{\infty} \left( \sum_{k=0}^{n} \binom{r+k}{l} \binom{n-k}{j} \right) z^n
\end{align*}
Therefore, we have an identity
\begin{align*}
    \sum_{k=0}^{n} \binom{r+k}{l} \binom{n-k}{j} = \coefficientOf{z}{n} \frac{z^{l+j-r}}{(1-z)^{l+j+2}}
\end{align*}
Having $l=r$ and $j=0$
\begin{align*}
    \sum_{k=0}^{n} \binom{r+k}{r} \binom{n-k}{0} = \coefficientOf{z}{n} \frac{1}{(1-z)^{r+2}}
\end{align*}
Which evaluates
\begin{align*}
    \binom{r+n-1}{n}   &= \coefficientOf{z}{n} \frac{1}{(1-z)^r} = \coefficientOf{z}{n} \sum_{k=0}^\infty \binom{r+k-1}{k} z^k \\
    \binom{r+n-1}{r-1} &= \coefficientOf{z}{n} \frac{1}{(1-z)^r} = \coefficientOf{z}{n} \sum_{k=0}^\infty \binom{r+k-1}{k} z^k
\end{align*}
So that
\begin{align*}
    \coefficientOf{z}{n} \frac{1}{(1-z)^{r+2}} = \binom{r+n+1}{r+1}
\end{align*}
Therefore, desired result holds
\begin{align*}
    \sum_{k=0}^{n} \binom{r+k}{r} = \binom{r+n+1}{n}
\end{align*}
\subsection{Right-hand side of the identity}
\label{subsec:right-hand-side-of-the-identity}
Consider the binomial coefficient $\binom{r+n+1}{r+1}$ from RHS of the problem~\eqref{eq:identity-to-prove-problem-5}
and its generating function.
First, let's do not forget about symmetry rule
\begin{align*}
    \binom{r+n+1}{r+1} = \binom{r+n+1}{n}
\end{align*}
So it is the coefficient of $z^n$ in the generating function
\begin{align*}
    \binom{(r+1)+n}{(r+1)} = \coefficientOf{z}{n} \sum_{k=0}^{\infty} \binom{(r+1)+k}{(r+1)} z^k
\end{align*}
Setting $c=r+1$ and $u=r+1$ we have, fixed $c,u$
\begin{align*}
    \binom{c+n}{u} = \coefficientOf{z}{n} \sum_{k=0}^{\infty} \binom{c+k}{u} z^k
\end{align*}
Then working generating function is
\begin{align*}
    F_{c,u}(z) = \sum_{m=0}^{\infty} \binom{c+m}{u} z^{m} = \frac{z^u}{(1-z)^{u+1} \cdot z^c}
\end{align*}
So the right-hand side in terms of $F$ is
\begin{align*}
    \binom{(r+1)+n}{(r+1)}
    &= \coefficientOf{z}{n} F_{r+1, r+1}(z)
    = \coefficientOf{z}{n} \frac{z^{r+1}}{(1-z)^{r+1+1} \cdot z^{r+1}} \\
    &= \coefficientOf{z}{n} \frac{1}{(1-z)^{r+2}} \\
\end{align*}





\clearpage
