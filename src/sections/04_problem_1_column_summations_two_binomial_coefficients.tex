\section{Problem 1: Prove that $\sum_{k=0}^{t} \binom{t-k}{r} \binom{k}{s} = \binom{t+1}{r+s+1}$}
\label{sec:problem-1-column-summation-of-two-binomial-coefficients-mse}
Prove that
\begin{equation}
    \sum_{k=0}^{t} \binom{t-k}{r} \binom{k}{s} = \binom{t+1}{r+s+1}
    \label{eq:identity-to-prove-problem-1}
\end{equation}

We can see that iteration in the left-hand side of the equation~\eqref{eq:identity-to-prove-problem-1} is running
over the upper index of binomial coefficients, so let's figure out proper generating function for it.
We know that
\begin{equation}
    \sum_{n=0}^{\infty} \binom{n}{k} y^n = \frac{y^k}{(1-y)^{k+1}}\label{eq:generating-function-base}
\end{equation}
Now we keep attention to the lower index of binomial coefficient in right-hand side
of the equation~\eqref{eq:identity-to-prove-problem-1} which is $r+s+1$.
We have to match some generating function to reach the $\sum_{n=0}^{\infty} \binom{n}{r+s+1} x^n$
Let be generating functions
\begin{align*}
    A_r(x) &= \sum_{l=0}^{\infty} \binom{l}{r} x^l\\
    B_s(x) &= \sum_{k=0}^{\infty} \binom{k}{s} x^k
\end{align*}
Now let's match our generating function
\begin{align*}
    x A_r(x) B_s(x)
    &= x \cdot \left( \sum_{l=0}^{\infty} \binom{l}{r} x^l \right) \cdot \left( \sum_{k=0}^{\infty} \binom{k}{s} x^k \right) \\
    &= x \cdot \frac{x^r}{(1-x)^{r+1}} \cdot \frac{x^s}{(1-x)^{s+1}} = \frac{x^{r+s+1}}{(1-x)^{r+s+2}} \\
    &= \sum_{n=0}^{\infty} \binom{n}{r+s+1} x^n
\end{align*}
Because of the equation~\eqref{eq:generating-function-base}.
Actually, we could simply substitute $k=r+s+1$ to the equation~\eqref{eq:generating-function-base} to reach desired
generating function, anyway.

The coefficient of $x^{t+1}$ in the $x \cdot \sum_{l=0}^{\infty} \binom{l}{r} x^l \cdot \sum_{k=0}^{\infty} \binom{k}{s} x^k$ is
\begin{align*}
[x^{t+1}]
    x A_r(x) B_s(x) = \sum_{k=0}^{t} a_k b_{t-k} = \sum_{k=0}^{t} \binom{t-k}{r} \binom{k}{s}
\end{align*}
Note that upper summation bound is $t$ while coefficient is $[z^{t+1}]$ it is because of the $x$ factor in the generating function.
General rule for that is
\begin{align*}
[z^{p-q}]
    A(z)=[z^p]z^qA(z)
\end{align*}
While
\begin{align*}
[x^{t+1}]
    \sum_{n=0}^{\infty} \binom{n}{r+s+1} x^n = \binom{t+1}{r+s+1}
\end{align*}
Thus
\begin{align*}
    \binom{t+1}{r+s+1} = \sum_{k=0}^{t} \binom{t-k}{r} \binom{k}{s}
\end{align*}

This approach is based on generating functions
by \url{http://math.arizona.edu/~faris/combinatoricsweb/generate.pdf}

\subsection{Flow of the solution}
\label{subsec:flow-of-the-solution}
\begin{itemize}
    \item First, keep your attention to the left part of the problem~\eqref{eq:identity-to-prove-problem-1}
    to see over what index iteration is running, there can be three cases: lower, upper, or both.
    If both find identity to simplify it to run over either lower or upper index.
    If is upper index use generating function~\eqref{eq:generating-function-base}.
    If lower index use binomial theorem as generating function.
    \item Then keep your attention to the right part of the problem~\eqref{eq:identity-to-prove-problem-1},
    precisely the lower index of binomial coefficient, as it defines the generating function
    we're looking for.
    \item Then seeing the sum over binomial coefficients multiplication,
    it means that it should be expressed in terms of convolution of two generating functions,
    for our case is $[x^{t+1}] x A_r(x) B_s(x)$.
    \item Match the coefficient given by the convolution of two generating functions and the generating function.
\end{itemize}


\clearpage
